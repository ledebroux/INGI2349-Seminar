\documentclass[11pt,a4paper]{article}
\usepackage[utf8]{inputenc}
\usepackage[left=2.00cm, right=2.00cm, top=2.00cm, bottom=2.00cm]{geometry}
\author{Olivier Tilmans, Leonard Debroux}
\title{Failures in SDN}
\begin{document}

\section{Reference articles}
\subsection{Verifying forwarding plane connectivity}
This article presents a way to detect and locate a single arbitrary node or link failure.
In order to do so, the controller will install multiple sets of static routing rules in the switches.

The first set is used to detect whether there is a failure or not in the network.
The main idea is to create an Euler cycle in the network ()if the network does not allow one, the trick is to duplicate all link and build a directed graph, which will force the existence of such a cycle). The controller will then attach itself to a node in the network and send a control message. This message will be then propagated around the cycle and will loop back to the controller if all links and node are operating properly.

The second set of forwarding rules is used to locate the failure in the network. The controller will send a second control message which should loop back to the control after reaching a given node. The controller will keep repeating this operation, allowing it to perform a binary search over the nodes until it locates the link or node that has failed.

\textbf{Limitations:}
\begin{itemize}
\item This only works in case of a single link or node failure. (As the same node can be in different places fo the cycle thus breaking it multiple times)
\end{itemize}
\subsection{FatTire: Declarative Fault Tolerance for SDN}

\subsection{Automatic failure recovery for SDN}


\subsection{Enabling fast failure recovery in OpenFlow Networks}


\end{document}