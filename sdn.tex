\documentclass[11pt,a4paper]{article}
\usepackage[utf8]{inputenc}
\usepackage[left=2.00cm, right=2.00cm, top=2.00cm, bottom=2.00cm]{geometry}
\author{Olivier Tilmans, Leonard Debroux}
\title{Failures in SDN}
\begin{document}

\section{Reference articles}
\subsection{Verifying forwarding plane connectivity}
This article present a way to detect and locate a single arbitrary node or link failure.
In order to do so, the controller will install multiple set of static routing rules in the switches.

The first
The main idea is to create an Euler cycle in the network, if the network does not allow one, the trick is to 
\subsection{FatTire: Declarative Fault Tolerance for SDN}

\subsection{Automatic failure recovery for SDN}

\subsection{Enabling fast failure recovery in OpenFlow Networks}
The topic of this article is to present a way of doing fast failure recovery in openflow networks.
The idea is to make sure that upon the acknowledgment of a failure by the controller, there is an efficient way to quickly inform the openflow switches.

To do so, the article presents an algorithm on how to recover from a failure fast enough and on the whole network.
For the method to work, there are several requierments from the controller. These are the following:
\begin{itemize}
	\item The controller must know about the failure
	\item The controller must remember the paths it established
	\item The controller must be able to compute a new path
	\item The controller knows all the current flows in the network
\end{itemize}
The algorithm states that upon a link failure, a new path is computed for every path that goes through this link.
Then, in each switch of the network that has an entry related to the old broken path, the entry is delete and new ones are pushed to establish the new path.

The results in the article are restoration times of about 12ms. In large networks, the goal is to propose restoration times that are below 50ms. 
\end{document}