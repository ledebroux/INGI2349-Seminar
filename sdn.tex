\documentclass[11pt,a4paper]{article}
\usepackage[utf8]{inputenc}
\usepackage[left=2.00cm, right=2.00cm, top=2.00cm, bottom=2.00cm]{geometry}
\author{Olivier Tilmans, Leonard Debroux}
\title{Failures in SDN}
\begin{document}
\maketitle
\section{Reference articles}
\subsection{Verifying forwarding plane connectivity}
This article presents a way to detect and locate a single arbitrary node or link failure.
In order to do so, the controller will install multiple sets of static routing rules in the switches.

The first set is used to detect whether there is a failure or not in the network.
The main idea is to create an Euler cycle in the network ()if the network does not allow one, the trick is to duplicate all link and build a directed graph, which will force the existence of such a cycle). The controller will then attach itself to a node in the network and send a control message. This message will be then propagated around the cycle and will loop back to the controller if all links and node are operating properly.

The second set of forwarding rules is used to locate the failure in the network. The controller will send a second control message which should loop back to the control after reaching a given node. The controller will keep repeating this operation, allowing it to perform a binary search over the nodes until it locates the link or node that has failed.

\textbf{Limitations:}
\begin{itemize}
	\item This only works in case of a single link or node failure. (As the same node can be in different places fo the cycle thus breaking it multiple times)
\end{itemize}

\subsection{FatTire: Declarative Fault Tolerance for SDN}
This paper introduces a new language constructs that enable a programmer to specify flow invariant, even in case of failure. This construction is then compiled in order to generate backup forwarding table entries using OpenFlow fast failover mechanism and installed on the switches.
The paper describes the language syntax as well as its implementation.

\textbf{Limitations:}
\begin{itemize}
	\item This will require switches to have huge forwarding tables if it wants to account for every possible failure (if the programmer enabled FatTire for all possible flows).
\end{itemize}

\subsection{Automatic failure recovery for SDN}
This paper advocates the need to separate the failure recovery mechanism on the controller from the forwarding and policy logic. It presents such a framework which is built for the POX controller, and provides a simple module that will handle failures and recover from them allowing developer to write failure-agnostic code.

This framework works in two phases:
\begin{enumerate}
	\item When there are no failures in the network, it records all events that happen on the controller.
	\item When a failure is detected, it performs the recovery int wo phases:
		\begin{description}
			\item[The replay phase] It creates a clean copy of the original controller, minus the failed nodes, and replay all events it has recorded, which leads to a new coherent controller state accounting for the failure.
			\item[The reconfiguration phase] It pushes the new network state to the controller and the switches
		\end{description}
\end{enumerate}

\textbf{Limitations:}
\begin{itemize}
	\item This is a "slow" process, as the configuration change has to be pushed from the controller to the switches after being signaled to the controller
\end{itemize}

\subsection{Enabling fast failure recovery in OpenFlow Networks}
The topic of this article is to present a way of doing fast failure recovery in openflow networks.
The idea is to make sure that upon the acknowledgment of a failure by the controller, there is an efficient way to quickly inform the openflow switches.

To do so, the article presents an algorithm on how to recover from a failure fast enough and on the whole network.
For the method to work, there are several requierments from the controller. These are the following:
\begin{itemize}
	\item The controller must know about the failure
	\item The controller must remember the paths it established
	\item The controller must be able to compute a new path
	\item The controller knows all the current flows in the network
\end{itemize}
The algorithm states that upon a link failure, a new path is computed for every path that goes through this link.
Then, in each switch of the network that has an entry related to the old broken path, the entry is delete and new ones are pushed to establish the new path.

The results in the article are restoration times of about 12ms. In large networks, the goal is to propose restoration times that are below 50ms. 

\subsection{Scalable Fault Management for OpenFlow}
Treats about \textbf{failure detection}. \textit{didnt read all of it yet}

This paper present an alternative solution to LLDP to monitor the network health. To do so, the switches are extended to be able to send and to react to OAM (Operation, Administration, and Maintenance) packets.

The advantage taken from this solution is that the controller does not have to worry about monitoring the network for failures anymore. If it had to do it, to be able to react within a certain limit of time (50ms), it would mean send a high amount of LLDP packets per second, per link and per end-to-end tunnel. This causes serious scalability issues in openflow.

In this solution, if a failure occurs, it is detected without the need of the packet to get back to the sender. When a OAM packet is received on the egress side of a tunnel, the switch updates a state accordingly to the packet. For each tunnel monitored at the switch, there is a timer that is updated for each monitoring packet received. If that timer expires, the switch notifies the controller that the tunnel has failed.

\subsection{Network Architecture for joint Failure Recovery and Traffic Engineering}
This paper presents a way to unify load balancing and failure recovery.

It first starts by presenting this unified approach in a protocol agnostic way, then presents some experimental results and finally concludes by providing a few hints on how to deploy it in real networks (i.e. by using MPLS, Openflow, \ldots).

The main idea is that routers should be simple forwarding switches, much like in the SDN paradigm. An external management system precomputes multiple paths for each pair of edge routers which have traffic between them. The routers in the network will them perform load balancing by splitting the traffic over the normal paths and the backup paths. If a path fails, the routers will simply remove the failed path from their FIB, and continue forwarding traffic over the remaining paths.

In this system, failures are detected at a path level, by the ingress/egress routers, which will react to the failures themselves by rebalancing the traffic on the remaining path, ignoring the load informations. As this recovery is perform without this information, the edge routers do not need to broadcast real-time updates about the network state, which means that the reaction of a given router to a failure is deterministic. 
\end{document}